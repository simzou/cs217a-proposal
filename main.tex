%%%%%%%%%%%%%%%%%%%%%%%%%%%%%%%%%%%%%%%%%%%%%%%%%%%%%%%%%%%%%%%%%%%%%%%%%%%%%%%%
%2345678901234567890123456789012345678901234567890123456789012345678901234567890
%        1         2         3         4         5         6         7         8

\documentclass[letterpaper, 10 pt, conference]{ieeeconf}  % Comment this line out
                                                          % if you need a4paper
%\documentclass[a4paper, 10pt, conference]{ieeeconf}      % Use this line for a4
                                                          % paper

\IEEEoverridecommandlockouts                              % This command is only
                                                          % needed if you want to
                                                          % use the \thanks command
\overrideIEEEmargins
% See the \addtolength command later in the file to balance the column lengths
% on the last page of the document

\usepackage{authblk}

% The following packages can be found on http:\\www.ctan.org
%\usepackage{graphics} % for pdf, bitmapped graphics files
%\usepackage{epsfig} % for postscript graphics files
%\usepackage{mathptmx} % assumes new font selection scheme installed
%\usepackage{times} % assumes new font selection scheme installed
%\usepackage{amsmath} % assumes amsmath package installed
%\usepackage{amssymb}  % assumes amsmath package installed

\title{\LARGE \bf
Proposal: VectorChat - Chat Application over NDN
}

% \author{ \parbox{3 in}{\centering Huibert Kwakernaak*
%         \thanks{*Use the $\backslash$thanks command to put information here}\\
%         Faculty of Electrical Engineering, Mathematics and Computer Science\\
%         University of Twente\\
%         7500 AE Enschede, The Netherlands\\
%         {\tt\small h.kwakernaak@autsubmit.com}}
%         \hspace*{ 0.5 in}
%         \parbox{3 in}{ \centering Pradeep Misra**
%         \thanks{**The footnote marks may be inserted manually}\\
%        Department of Electrical Engineering \\
%         Wright State University\\
%         Dayton, OH 45435, USA\\
%         {\tt\small pmisra@cs.wright.edu}}
% }

\begin{document}
\author{Howard Huang}
\author{Ricky Lee}
\author{Simon Zou}
\affil{\textit {\{howardhuang,ricky.lee,simonzou\}@ucla.edu}}

\title{\LARGE \bf
Proposal: VectorChat - Chat Application over NDN
}
\maketitle


%%%%%%%%%%%%%%%%%%%%%%%%%%%%%%%%%%%%%%%%%%%%%%%%%%%%%%%%%%%%%%%%%%%%%%%%%%%%%%%%
\begin{abstract}

We propose a new chat application over Named Data Networking (NDN), called VectorChat. The application will utilize the implementation of VectorSync to store and send data between users and also use ideas from Audio Conference Tool (ACT) to maintain authenticity and confidentiality between chat participants.

\end{abstract}

%%%%%%%%%%%%%%%%%%%%%%%%%%%%%%%%%%%%%%%%%%%%%%%%%%%%%%%%%%%%%%%%%%%%%%%%%%%%%%%%
\section{PROPOSED DESIGNS}

\subsection{Functionality}

We will take inspiration from some of the most successful and widely used chat applications today, such as Slack, Discord, and Facebook Messenger, which in turn are spiritual successors to IRC. We propose a chat application with the following abstractions:

\begin{itemize}
\item \textbf{Users} - entities that can send, read, and receive messages
\item \textbf{Rooms} - the channels across which messages are sent. Upon creation, they are are uniquely identified by the users or a unique name. There are several different cases of rooms. Rooms support the inviting and removing of users. Group chats and direct messages can become channels if given a name, but not vice versa.
\begin{itemize}
\item \textbf{Direct Messages} - A room between $n = 2$ people
\item \textbf{Group chats} - A room with $n > 2$ people, uniquely identified by the participants
\item \textbf{Channel} - A room with $n \geq 1$ people, uniquely identified by a name 
\end{itemize}
\item \textbf{Messages} - the actual content of messages, uniquely identified by sender (via digital signature), the room they sent it to, and the time-stamp. 
\end{itemize}

Take it all together and the data we’re trying to sync is something I think akin to a list of message objects, i.e: 
\begin{verbatim}
rid_1,ts_1,dsm_1
rid_1,ts_2,dsm_2
rid_2,ts_3,dsm_3
rid_3,ts_4,dsm_4
rid_2,ts_2,dsm_5
\end{verbatim}

where \begin{verbatim}
rid = room_id, 
ts = timestamp, 
dsm = digitally_signed(message)
\end{verbatim}
Our job would be to make sure this works properly in the sense that this list gets updated correctly. The fact that we don’t have to build a UI means that it doesn’t actually have to be all that coherent, the front end is what would be responsible for displaying rooms in the right places and the messages in the correct order and so forth. We just have to make sure information can be exchanged, synced, and that certain constraints are obeyed (e.g. can’t read or send messages in rooms you are not a part of). 

\subsection{Security}

	The chat application will model its security after ACT, an audio conferencing tool over NDN [2].
	The user who starts a message on the application will be considered the \textbf{Organizer} of the chat. The \textbf{Organizer} is responsible for adding and removing participants to the chat room. This is done through an encryption-based access control scheme, which allow the eligible participants of the chat to communicate. As explained in the ACT paper, the \textbf{Organizer} will distribute a public key to all participants and the the \textbf{Organizer} will encrypt all messages in the chat with his private key.

\section{PROJECT TIMELINE}

\subsection{High Level Todo List}

\begin{itemize}
\item Read VectorSync paper.
\item Read and get VectorSync code [1] up and running.
\item Implement classes and logic based on the design described above using VectorSync and NDN (e.g. a \verb|Room|, \verb|User| class, etc)
\end{itemize}

\subsection{Project Specific API Design}

\verb|User| class public API (work-in-progress)
\begin{itemize}
\item \begin{verbatim}create_new_room (one_other_user)\end{verbatim} 
\item \begin{verbatim}create_new_room (multiple_users)\end{verbatim} 
\item \begin{verbatim}create_new_room (channel_name)\end{verbatim}
\item \begin{verbatim}send_message (room_id, message_content)\end{verbatim}
\item \begin{verbatim}join_channel (channel_name)\end{verbatim}
\item \begin{verbatim}exit_channel (channel_name)\end{verbatim}
\item \begin{verbatim}invite_to_join_channel (user, channel_name)\end{verbatim}
\end{itemize}

Private methods for distributing public keys would occur in the \verb|create_new_room| methods. 

\verb|Room| class public API (work-in-progress)
\begin{itemize}
\item \begin{verbatim}add_user (user)\end{verbatim} 
\item \begin{verbatim}remove_user (user)\end{verbatim} 
\item \begin{verbatim}rename (channel_name)\end{verbatim}
\end{itemize}

\verb|Message| struct
\begin{itemize}
\item \begin{verbatim}room_id\end{verbatim} 
\item \begin{verbatim}timestamp\end{verbatim} 
\item \begin{verbatim}message_content\end{verbatim}
\end{itemize}

\subsection{Proposed Project Schedule}
\begin{verbatim}
2/16 - Friday, Week 06 - Proposal Submitted
2/23 - Friday, Week 07 - VectorSync Code example code runs
                         Paper references read
                         Begin Implementing API design
3/02 - Friday, Week 08 - Finish Implementing API 
                         Add security via NDN modules     
3/09 - Friday, Week 09 - Unit tests, Finish rough draft of report
3/16 - Friday, Week 10 - Present project

Stretch goal / continuing work: Begin UI application 
\end{verbatim}
\addtolength{\textheight}{-12cm}   % This command serves to balance the column lengths
                                  % on the last page of the document manually. It shortens
                                  % the textheight of the last page by a suitable amount.
                                  % This command does not take effect until the next page
                                  % so it should come on the page before the last. Make
                                  % sure that you do not shorten the textheight too much.

%%%%%%%%%%%%%%%%%%%%%%%%%%%%%%%%%%%%%%%%%%%%%%%%%%%%%%%%%%%%%%%%%%%%%%%%%%%%%%%%

\begin{thebibliography}{99}

\bibitem{c1} A survey of Distributed Dataset Synchronization in Named Data Networking: https://named-data.net/wp-content/uploads/2017/05/ndn-0053-1-sync-survey.pdf, Source Code: https://github.com/wentaoshang/VectorSync.

\bibitem{c2} A New Approach to Securing Audio Conference Tools: http://sprout.ics.uci.edu/projects/ndn/papers/awfit-audiosec.pdf

\end{thebibliography}

\end{document}
